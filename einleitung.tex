\section{Einleitung} % (fold)
\label{sec:einleitung}

\subsection{Begründung der Problemstellung} % (fold)
\label{sub:begrundung_der_problemstellung}
% subsection begrundung_der_problemstellung (end)

Online--Plattformen nehmen eine zentrale Rolle im Alltag von Studierenden ein. An Fernhochschulen werden hier zunächst die organisatorischen Aufgaben wie die An- und Abmeldungen zu Präsenzseminaren und Prüfungen und die Notenbekanntgabe abgewickelt. Darüber hinaus werden aber auch immer mehr Aufgaben der Wissensvermittlung und des Lernens über das Internet wahrgenommen. Hierzu stehen eine große Auswahl an Kommunikatinsformen zur Verfügung, welche für unterschiedliche Aspekte des Lernens genutzt werden können.

Internetforen gehören zu den ältesten Werkzeugen des Web 2.0 und ermöglichen es Gruppen, sich über Lerninhalte auszutauschen. Hierbei ist es auch möglich, dass das Wissen nicht nur von den Dozenten zu den Studierenden weitergegeben wird, sondern die Studierenden können sich auch gegenseitig Fragen beantworten und gegebenenfalls gemeinsam Lösungen erarbeiten. Hierbei ist eine Gliederung in verschieden große Organisationseinheiten\footnote{z.B. in modulspezifische Foren, studiengangs- oder studienbereichspezifische Foren bis hin zum Austausch mit allen eingeschreibenen Studierenden der Hochschule.} möglich.

Ein solches Kommunikationsangebot ist sorgfältig mit den anderen Elementen des Studiums, sowohl online als auch offline, abzustimmen.

\subsection{Ziele dieser Arbeit} % (fold)
\label{sub:ziele_dieser_arbeit}
\textbf{Ziel dieser Arbeit ist es, ein Forenkonzept für die Lernplattform einer Fernhochschule zu erstellen, welches das Lernen und den Studienablauf unterstützt.}

Hierzu werden im Kapitel~\myref{sec:kommunikationsformen_im_internet} zunächst die verschiedenen Möglichkeiten der Kommunikation im Internet betrachtet und anschließend werden in den Kapiteln~\myref{sec:ziele} und \myref{sec:technische_und_wirtschaftliche_aspekte} die Anforderungen an das Forensystem sowie dessen Anforderungen an Budget und Technik ermittelt. Eine Betrachtung von verschiedenen Beispielen erfolgreicher Forensysteme schließt in Kapitel~\myref{sec:best_practice} den Grundlagenteil dieser Arbeit ab.

Nach einer kurzen Vorstellung der bereits ausgewählten Forensoftware in Kapitel~\myref{sec:beschreibung_der_forensoftware} werden in den folgenden Kapiteln Empfehlungen für die Benutzer- und Themenstruktur sowie der umzusetzenden Funktionalität erarbeitet. In den Kapiteln~\myref{sec:redaktionelles_konzept} und \myref{sec:controlling} folgen Empfehlungen für die aktive inhaltliche Gestaltung sowie für die Messung der Zielerreichung.
% subsection ziele_dieser_arbeit (end)

\subsection{Methodik} % (fold)
\label{sub:methodik}
Literaturrecherche, Expertenbefragungen, evt. Onlineumfrage unter Studierenden
% subsection methodik (end)

\subsection{Abgrenzung} % (fold)
\label{sub:abgrenzung}
Pädagogischer Nutzen und Notwendigkeit wird vorausgesetzt und nicht explizit untersucht.

% subsection abgrenzung (end)

% section einleitung (end)