\section{Einleitung} % (fold)
\label{sec:einleitung}

\subsection{Begründung der Problemstellung} % (fold)
\label{sub:begrundung_der_problemstellung}
% subsection begrundung_der_problemstellung (end)

Online--Plattformen nehmen eine zentrale Rolle im Alltag von Studierenden ein. An Fernhochschulen werden hier zunächst die organisatorischen Aufgaben wie die An- und Abmeldungen zu Präsenzseminaren und Prüfungen und die Notenbekanntgabe abgewickelt. Darüber hinaus werden aber auch immer mehr Aufgaben der Wissensvermittlung und des Lernens über das Internet wahrgenommen. Hierzu stehen eine große Auswahl an Kommunikatinsformen zur Verfügung, welche für unterschiedliche Aspekte des Lernens genutzt werden können.

Internetforen gehören zu den ältesten Werkzeugen des Web 2.0\footnote{Die Wurzeln der Internet–Diskussionsforen liegen im Usenet und reichen somit in die Zeiten vor dem \ac{WWW} zurück.} und ermöglichen es Gruppen, sich über Lerninhalte auszutauschen. Hierbei ist es auch möglich, dass das Wissen nicht nur von den Dozenten zu den Studierenden weitergegeben wird, sondern die Studierenden können sich auch gegenseitig Fragen beantworten und gegebenenfalls gemeinsam Lösungen erarbeiten. Hierbei ist eine Gliederung in verschieden große Organisationseinheiten\footnote{z.B. in modulspezifische Foren, studiengangs- oder studienbereichspezifische Foren bis hin zum Austausch mit allen eingeschreibenen Studierenden der Hochschule.} möglich. Gleichzeitig können sich die  Studierendengruppen verkleinern imd individualisieren, was das Lernen intensiver und den Austausch besser macht.\footnote{\cite{roegner:trends}, ab 2'23"}

Ein solches Kommunikationsangebot ist sorgfältig mit den anderen Elementen des Studiums, sowohl online als auch offline, abzustimmen.

\subsection{Ziele dieser Arbeit} % (fold)
\label{sub:ziele_dieser_arbeit}
\textbf{Ziel dieser Arbeit ist die Konzeption einer Plattform für gemeinschaftliches Lernen zur Ergänzung des Online Campus Systems einer Fernhochschule.}

Zur Erstellung eines solchen Konzepts sollen die theoretischen Grundlagen des gemeinschaftlichen Computerunterstützten Lernen zusammengefasst werden sinnvolle und erreichbare Ziele für die Plattform formulieren zu können. Ebenfalls ist die Einbindung in die Organisation der Fernhochschule und in den Markt zu betrachten. Die wesentlichen Bestandteile der Kostenstruktur sowie Möglichkeiten die Zielerreichung zu messen sind zu benennen.

Hierzu werden zunächst in Kapitel~\myref{sec:definitionen} die notwendigen Begriffe definiert und in Kapitel~\myref{sec:werkzeuge_des_cscl} die entsprechenden Werkzeuge beschrieben. In Kapitel~\myref{sec:ziele} und \myref{sec:technische_und_wirtschaftliche_aspekte} werden mögliche Bereiche genannt, die für die Erarbeitung von Zielen relevant sind. Zum Abschluss des Grundlagenteils im Kapitel~\myref{sec:best_practice} werden einige Lernangebote im Internet auf Elemente hin untersucht, die einer Lernplattform zum Erfolg verhelfen können.

Wie schon die Kapitel~\ref{sec:definitionen} bis \ref{sec:best_practice} allgemein gehalten sind, sind auch die folgenden Kapitel für viele Lernplattformen anwendbar. Es werden jedoch auf Umstände und Eigenheiten des Studiums und der bestehenden Organisation an der AKAD–University, Stuttgart als Rahmen hinzugezogen, falls dies wie etwa in Kapitel~\myref{sec:plazierung_am_markt} zur Konkretisierung der Aussagen hilfreich ist. In den Kapiteln~\ref{sec:technische_und_wirtschaftliche_aspekte} bis \ref{sec:redaktionelles_konzept} werden Empfehlungen für Struktur, Qualitätssicherung und zum redaktionellen Konzept gegeben. Den Abschluss bildet die Erarbeitung von Möglichkeiten zur Messung der Zielerreichung im Kapitel~\myref{sec:controlling}.
% subsection ziele_dieser_arbeit (end)

\subsection{Methodik} % (fold)
\label{sub:methodik}
Für den Grundlagenteil dieser Arbeit wurde als Methodik die Literaturrecherche gewählt. Ebenfalls wurden bestehende Lernplattformen, die der Autor zum aus eigener Lernerfahrung kennt auf typische Elemente sowie auf besonders gelungene Herangehensweisen hin untersucht. Im Hauptteil wurden dann durch Kombination der Elemente des Grundlagenteil, der Erfahrungen des Autors als Studierender an der AKAD–University und von Elementen der Managementlehre die notwendigen Teilkonzepte erarbeitet, welche in Kombination das angestrebte Gesamtkonzept ergeben.
% subsection methodik (end)

\subsection{Abgrenzung} % (fold)
\label{sub:abgrenzung}
Der pädagogischer Nutzen und Notwendigkeit wird vorausgesetzt und nicht explizit untersucht. Ebenso sind die organisatorischen und betriebswirtschaftlichen Ziele allgemein gehalten. Es wird davon ausgegangen, dass unabhängig von der vorhandenen Qualität und Effizienz immer noch eine Steigerung möglich und wünschenswert ist. Das erstellte Konzept orientiert sich zwar an den Strukturen und Gegebenheiten der AKAD–University, die beschriebenen Anforderungen sind jedoch allgemein gehalten, so dass sie problemlos auf andere Lernplattformen übertragen lassen. 

Beim Erstellen des redaktionellen Konzepts in Kapitel~\myref{sec:redaktionelles_konzept} liegt das Hauptaugenmerk auf den Werkzeugen, bei denen für Studierende die größten Gestaltungsmöglichkeiten liegen: Dem Forum und dem Wiki. Auf eine konkrete Ausformulierung eines pädagogischen Teilkonzepts wird in dieser Arbeit verzichtet, auch wenn einzelne, für das Gesamtsystem wichtige Aspekte erwähnt werden.
% subsection abgrenzung (end)

% section einleitung (end)