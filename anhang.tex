\appendix
\section*{Anhang}
\addcontentsline{toc}{section}{Anhang}


\subsection*{Anhang 1: Antwortzeiten im Forum ERP4students} % (fold)
\label{sub:antwortzeiten_im_forum_erp4students}
\addcontentsline{toc}{subsection}{Anhang 1: Antwortzeiten im Forum ERP4students}

Die in Tabelle~\ref{tab:erp4students} aufgeführten Werte sind Stichproben der Frage– und Antwortzeitpunkte im Forum des Fernlehrgangs „ERP4students — Einführung in SAP BW“ der von der Universität Duisburg–Essen im Wintersemester 2014/2015 durchgeführt wurde. In den Kursbegleitenden Unterforen „Teil 0-0: Forum zur Einführung“. „Teil I-1: Grundlegende Datenstrukturen im Data Warehouse Layer” sowie „Teil II-1: SAP BW Query Design“ wurden die Zeitstempel der Fragen der Studierenden sowie die der Dozentenantworten erfasst und die zwischen Frage und Antwort verstrichene Zeit ermittelt. Die Unterforen wurden ausgewählt, um alle Lernphasen (Orientierung, Grundlagen, Anwendungen) sowie den gesamten Kurszeitraum von 15. November 2014 bis 15. März 2015 abzudecken.

Von den in den drei Unterforen 53 gestellten Fragen von Studierenden wurden 22 innerhalb von 15 Minuten beantwortet. Weitere 16 Antworten wurden in unter einer Stunde, weitere 11 in unter sechs Stunden beantwortet. Drei Fragen blieben zwischen sechs Stunden und einem Tag und lediglich eine einzige Antwort blieb länger als einen Tag unbeantwortet.\footnote{Diese Frage bildet einen extremen Ausreißer mit fast einer Woche Wartezeit.} Die Hälfte aller Fragen wurde in 20 Minuten oder schneller beantwortet.

%\begin{table}[H]
\begin{longtable}{lrrrr}
\caption{Antwortzeiten ERP4students} \\
\toprule
& \multicolumn{1}{l}{Frage} & \multicolumn{1}{l}{Antwort} & \multicolumn{2}{l}{Dauer}  \\ \midrule
\endfirsthead
\multicolumn{5}{@{}l}{ (…Fortsetzung)} \\
\toprule
& \multicolumn{1}{l}{Frage} & \multicolumn{1}{l}{Antwort} & \multicolumn{2}{l}{Dauer}  \\ \midrule
\endhead
\bottomrule
\multicolumn{5}{@{}r}{(Fortsetzung…)} \\
\endfoot
\bottomrule
\endlastfoot
Sa & 15.11.14 13:22 & 15.11.14 14:30 & 01:08:00 &  \\
Sa & 15.11.14 14:47 & 15.11.14 16:10 & 01:23:00 &  \\
Mo & 17.11.14 13:10 & 17.11.14 13:24 & 00:14:00 &  \\
Di & 18.11.14 12:50 & 18.11.14 13:01 & 00:11:00 &  \\
Di & 18.11.14 13:11 & 18.11.14 13:22 & 00:11:00 &  \\
Di & 18.11.14 19:58 & 18.11.14 20:21 & 00:23:00 &  \\
Di & 18.11.14 20:54 & 18.11.14 21:01 & 00:07:00 &  \\
Di & 18.11.14 21:20 & 18.11.14 21:34 & 00:14:00 &  \\
Fr & 21.11.14 11:20 & 21.11.14 11:33 & 00:13:00 &  \\
So & 23.11.14 16:03 & 23.11.14 16:16 & 00:13:00 &  \\
Di & 25.11.14 17:26 & 25.11.14 17:40 & 00:14:00 &  \\
Mi & 26.11.14 00:09 & 26.11.14 02:35 & 02:26:00 &  \\
Mi & 26.11.14 23:45 & 27.11.14 02:06 & 02:21:00 &  \\
Do & 27.11.14 10:08 & 27.11.14 12:12 & 02:04:00 &  \\
So & 30.11.14 16:33 & 30.11.14 16:42 & 00:09:00 &  \\
Mo & 01.12.14 20:05 & 01.12.14 20:57 & 00:52:00 &  \\
Sa & 06.12.14 14:30 & 06.12.14 14:43 & 00:13:00 &  \\
Sa & 06.12.14 15:05 & 06.12.14 15:22 & 00:17:00 &  \\
Sa & 06.12.14 17:10 & 06.12.14 17:21 & 00:11:00 &  \\
So & 14.12.14 22:52 & 21.12.14 11:57 & 157:05:00 &  \\
Mi & 17.12.14 21:49 & 17.12.14 21:58 & 00:09:00 &  \\
So & 21.12.14 12:58 & 21.12.14 13:54 & 00:56:00 &  \\
Sa & 27.12.14 16:18 & 27.12.14 16:46 & 00:28:00 &  \\
Sa & 27.12.14 17:26 & 27.12.14 19:12 & 01:46:00 & \footnote{Antwort wurde von einem Teilnehmer gegeben, der eine vorige Dozentenantwort erklärt hat.} \\
Mo & 19.01.15 22:15 & 20.01.15 07:07 & 08:52:00 &  \\
Sa & 24.01.15 15:46 & 24.01.15 16:04 & 00:18:00 &  \\
Di & 27.01.15 14:52 & 27.01.15 15:03 & 00:11:00 &  \\
Mo & 02.02.15 15:07 & 02.02.15 15:13 & 00:06:00 &  \\
Mo & 02.02.15 22:46 & 02.02.15 22:54 & 00:08:00 & \footnote{Frage wurde vom Fragesteller selbst beantwortet.} \\
Di & 03.02.15 02:11 & 03.02.15 02:19 & 00:08:00 &  \\
Fr & 06.02.15 20:45 & 06.02.15 22:10 & 01:25:00 &  \\
Sa & 07.02.15 23:40 & 08.02.15 00:00 & 00:20:00 &  \\
So & 08.02.15 01:37 & 08.02.15 10:40 & 09:03:00 &  \\
Mi & 11.02.15 08:54 & 11.02.15 09:32 & 00:38:00 &  \\
Mi & 11.02.15 16:20 & 11.02.15 16:40 & 00:20:00 &  \\
Fr & 20.02.15 01:19 & 20.02.15 08:50 & 07:31:00 &  \\
Fr & 20.02.15 11:01 & 20.02.15 11:13 & 00:12:00 &  \\
Sa & 21.02.15 23:34 & 21.02.15 23:49 & 00:15:00 &  \\
Sa & 21.02.15 23:55 & 22.02.15 02:24 & 02:29:00 &  \\
Do & 26.02.15 19:42 & 26.02.15 19:51 & 00:09:00 &  \\
Do & 26.02.15 19:57 & 26.02.15 22:03 & 02:06:00 &  \\
Mo & 02.03.15 15:22 & 02.03.15 17:21 & 01:59:00 &  \\
Mo & 09.03.15 16:49 & 09.03.15 17:15 & 00:26:00 &  \\
Mo & 09.03.15 17:00 & 09.03.15 17:17 & 00:17:00 &  \\
Mo & 09.03.15 17:19 & 09.03.15 21:05 & 03:46:00 &  \\
Mo & 09.03.15 17:20 & 09.03.15 17:33 & 00:13:00 &  \\
Do & 12.03.15 09:26 & 12.03.15 09:56 & 00:30:00 &  \\
Do & 12.03.15 10:20 & 12.03.15 10:41 & 00:21:00 &  \\
Do & 12.03.15 17:57 & 12.03.15 18:36 & 00:39:00 &  \\
Fr & 13.03.15 20:58 & 13.03.15 21:35 & 00:37:00 &  \\
So & 15.03.15 04:05 & 15.03.15 04:15 & 00:10:00 &  \\
So & 15.03.15 17:02 & 15.03.15 17:52 & 00:50:00 &  \\
So & 15.03.15 21:07 & 15.03.15 21:18 & 00:11:00 &  \\
\label{tab:erp4students}
\end{longtable}
%\end{table}
% subsection antwortzeiten_im_forum_erp4students (end)

\subsection*{Anhang 2: Vorschlag: Benutzerrechte} % (fold)
\label{sub:vorschlag_rollen_und_rechte}
\addcontentsline{toc}{subsection}{Anhang 2: Vorschlag: Benutzerechte}

Die folgenden Beschreibungen für die den einzelnen Rollen zugewiesenen Benutzerrechte sind als Beispiele zu verstehen und nicht vollständig:

\begin{table}[H]
\begin{center}
\begin{footnotesize}
\begin{tabular}{| l | C{1cm} | C{1cm} | C{1cm} | C{1cm} | C{1cm} |}  \hline                       
  \textbf{Aktionen} & 
	\begin{turn}{90}\textbf{Teilnehmer\vspace{0.1cm}}\end{turn} & 
	\begin{turn}{90}\textbf{Dozenten}\end{turn}  & 
	\begin{turn}{90}\textbf{Moderatoren}\end{turn} & 
	\begin{turn}{90}\textbf{Administratoren}\end{turn} \\ \hline 
	Beiträge erstellen					& X   & X   & X  & X    \\  \hline  
	Beiträge kommentieren		& X   & X   & X  & X    \\  \hline  
	Beiträge beantworten		& X   & X   & X  & X    \\  \hline  
	Beiträge bewerten		& X   & X   & X  & X    \\  \hline  
	Eigene Beiträge ändern/löschen		& X   & X   & X  & X    \\  \hline  
	Beiträge anderer ändern/löschen		&     &     & X  & X    \\  \hline  
	Unterforen (allg.) erstellen		&     &     & X  & X    \\  \hline  
	Unterforum (Lerngruppe) erstellen		&  X   & X    & X  & X    \\  \hline  
	Forenstruktur ändern	&     &     &   & X    \\  \hline  
\end{tabular}
\end{footnotesize}
\caption{Rollen und Rechte im Forum}
\label{tab:rundrforum}
\end{center}
\end{table}

Im Wiki sind die Berechtigungen dem Werkzeug entsprechend deutlich offener gestaltet:

\begin{table}[H]
\begin{center}
\begin{footnotesize}
\begin{tabular}{| l | C{1cm} | C{1cm} | C{1cm} | C{1cm} | C{1cm} |}  \hline                       
  \textbf{Aktion} & 
	\begin{turn}{90}\textbf{Teilnehmer\vspace{0.1cm}}\end{turn} & 
	\begin{turn}{90}\textbf{Dozenten}\end{turn}  & 
	\begin{turn}{90}\textbf{Moderatoren}\end{turn} & 
	\begin{turn}{90}\textbf{Administratoren}\end{turn} \\ \hline 
	Beiträge erstellen					& X   & X   & X  & X    \\  \hline  
	Beiträge bewerten		& X   & X   & X  & X    \\  \hline  
	Eigene Beiträge ändern/löschen		& X   & X   & X  & X    \\  \hline  
	Beiträge anderer ändern/löschen		& X   & X    & X  & X    \\  \hline  
	Unterseiten und Themen erstellen	& X    & X    & X  & X    \\  \hline  
	Wikistruktur ändern	& X    & X    & X  & X    \\  \hline  
\end{tabular}
\end{footnotesize}
\caption{Rollen und Rechte im Wiki}
\label{tab:rundrwiki}
\end{center}
\end{table}

Für die Wissensbasis ist eine Rechtevergabe ähnlich derer im Forum denkbar, jedoch sollten \textbf{alle Rechte} für \textbf{alle Rollen} nur unter dem Vorbehalt eines Reviews vergeben werden.


% subsection vorschlag_rollen_und_rechte (end)

% section anhang (end)

