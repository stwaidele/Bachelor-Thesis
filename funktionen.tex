%!TEX root = /Users/stwaidele/Dropbox (Leisinger)/02 - AKAD/Projektbericht/Möglichkeiten der Digitalen Kontaktaufnahme im Endkundenbereich/vorlage.tex

%% Definition for Codeschnipsel im Fließtext
\newcommand{\code}{\texttt}
% \newcommand{\buzz}{\textit}
\newcommand{\buzz}{\textit}

\newcommand{\todo}[1]{\fbox{\parbox{\textwidth}{\textbf{To do:} #1}}}
%\newcommand{\myref}[1]{„\ref{#1}~\nameref{#1}“}
\newcommand{\myref}[1]{\textit{\ref{#1}~\nameref{#1}}}


%% Für Codeblöcke mit Syntax-Highlighting
%% http://www.ctan.org/tex-archive/macros/latex/contrib/minted/
\usepackage{minted}
\definecolor{bg}{rgb}{0.95,0.95,0.95}

%% Feste Spaltenbreite
%% http://de.wikibooks.org/wiki/LaTeX-W%C3%B6rterbuch:_tabular
\usepackage{tabularx}
\newcolumntype{L}[1]{>{\raggedright\arraybackslash}p{#1}} % linksbündig mit Breitenangabe
\newcolumntype{C}[1]{>{\centering\arraybackslash}p{#1}} % zentriert mit Breitenangabe
\newcolumntype{R}[1]{>{\raggedleft\arraybackslash}p{#1}} % rechtsbündig mit Breitenangabe

%% http://texwelt.de/wissen/fragen/9144/text-um-90-grad-in-einer-tabelle-zelle-drehen
%% Alternative Lösung
\newcommand\tabrotate[1]{\rotatebox{90}{#1\hspace{\tabcolsep}}}
\newcommand\verschiebung[1][-.75\normalbaselineskip]{\vspace{#1}}
