\addcontentsline{toc}{section}{Schluss}

\section{Fazit und Ausblick} % (fold)
\label{sec:fazit_und_ausblick}


% section fazit_und_ausblick (end)

\subsection{Bücherkiste} % (fold)
\label{sub:bucherkiste}

Wird in der Abgabeversion entfernt.

„Weiterer wichtiger Punkt im Trend ist: Die neuen Studienformen, Lernformen. Es gibt jetzt erste Studienabschlüsse, komplett mit virtuellen Veranstaltungen, Onlinevorlesungen: Man ist nicht mehr verpflichtet in die Hochschule zu gehen. Das setzt natürlich eine gewisse Eignung voraus. Oder diese sogenannten Open Universities, diese MOOGs, diese Veranstaltungen die es jetzt gibt. Das heisst, sie können sich im Internet an Eliteuniversitäten, oder auch an der Kalaidos, verschiedene Veranstaltungen anschauen, dazu brauchen sie keine Zulassungsvoraussetzungen, die brauchen sie dann, wenn sie die Prüfung absolvieren und das Diplom in der Hand haben wollen. Bildung wird frei. Das geht sogar so weit, dass es sogenannte ICS gibt, das heisst die Studierendengruppen werden sich minimieren, dann ist Lernen intensiver, der Austausch ist viel besser möglich. Im Endeffekt kann das dann direkte Lerneinheit sein in dem  ein Student mit einem Professor Coaching macht, zu einem Thema.“\footnote{\cite{roegner:trends}, ab 2'23"}

\cite{meggle}\\
\cite{ojstersek}\\
\cite{schulmeister}\\


% subsection bucherkiste (end)