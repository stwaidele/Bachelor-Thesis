\section{Schluss} % (fold)
\label{sec:schluss}

\subsection{Fazit} % (fold)
\label{sub:fazit}
In der vorliegenden Arbeit wurden die für eine internetgestützte Plattform für gemeinschaftliches Lernen notwendigen Grundlagen erarbeitet. Dies beinhalt die theoretischen Grundlagen von \ac{CSCL}–Systemen, sowie deren konkrete Umsetzung in Form von Softwarewerkzeugen wie Wikis oder Diskussionsforen. Es wurden Zielbereiche für eine solche Plattform sowie die Methoden zur Erfolgskontrolle herausgearbeitet. Ein redaktionelles Konzept sowie Methoden zur Qualitätsicherung runden das Gesamtkonzept ab.

Das Ziel der Arbeit, die Erstellung einer Konzeption einer Plattform für gemeinschafltiches Lernen zur Ergänzung des Online–Campus–Portals einer Fernhochschule ist somit erreicht worden. Hierbei ist allerdings zu beachten, dass vor einer praktischen Umsetzung noch weitere Planungen notwendig und deutliche detailliertere Teilkonzepte zu erabeiten sind. 
% subsection fazit (end)

\subsection{Ausblick} % (fold)
\label{sub:ausblick}
Die in Kapitel~\myref{sec:best_practice} beschriebenen Aspekte basieren auf den in Kapitel~\myref{sec:ziele} formulierten Grundsätzen und Zielen. Sowohl die Ziele als auch die daraus abgeleiteten Methoden sollten in weiteren Arbeiten wissenschaftlich auf die pädagogischen Grundlagen und Effizienz untersucht werden. Ebenso sollte eine konkrete redaktionelle Planung der regelmäßigen Aktivitäten der Dozenten erstellt werden.

Die betriebswirtschaftlichen Grundlagen sollten auf Basis des tatsächlich verfügbaren Budget und der angestrebten Servicequalität erstellt werden. Ebenso ist die Untersuchung der am Markt angebotenen Softwarelösungen für \ac{CSCL}–Plattformen sowie deren Einzelkomponenten erforderlich, um eine fundierte Make–or–Buy Entscheidung treffen zu können. Hierzu ist auch die Erstellung einer detaillierten Leistungsbeschreibung notwendig.

Für die weitere Forschung bietet sich auch das Themenfeld „Semantik in CSCL–Plattformen“ an. Hier kann ermittelt werden, wie den einzelnen Beiträgen am besten Bedeutung zugewiesen werden kann und wie diese Bedeutung für den Lernvorgang gewinnbringend eingesetzt werden kann.
% subsection ausblick (end)

% section schluss (end)