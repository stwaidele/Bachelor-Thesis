\addcontentsline{toc}{section}{Grundlagen}
\label{sec:grundlagen}

\section{Kommunikationsformen im Internet} % (fold)
\label{sec:kommunikationsformen_im_internet}

\subsection{Kommunikationsmodell} % (fold)
\label{sub:kommunikationsmodell}
Kommunikation wird von \cite{shannonweaver} als Informationsübertragung beschrieben, die Zwischen einer Quelle  und einem Empfänger\footnote{In der Orginalskizze vom „Receiver“ also „Empfänger” und der „Destination“, also Ebenfalls „Empfänger” gesprochen. Aufgrund der Doppeldeutigkeit im Deutschen wird der „Receiver“ in dieser Arbeit stets als „Dekodierer“ bezeichnet.} mit Hilfe eines Übertragungsmediums stattfindet. Vor der Übertragung werden die Informationen vom Sender kodiert und nach der Übertragung vom Empfänger dekodiert.\footnote{vgl. \cite{shannonweaver}, Seite 34} Während im ursprünglichen, technischen Modell Störungen lediglich bei der Übertragung stattfinden, wurde das Modell auf die Kommunikation zwischen Menschen erweitert, bei der auch bei der Kodierung und Dekodierung Fehler auftreten können.

Bei der Kommunikation zwischen Menschen wird somit nach diesem Modell die Botschaft in Worte kodiert, akkustisch oder schriftlich übertragen, und anschließend wieder vom Empfänger dekodiert. Da bei schriftlicher Kommunikation im Vergleich zum persönlichen Gespräch Details wie Betonung und Gesichtsausdruck, Körperhaltung, etc. nicht übermittelt werden, besteht auch hier beim Kodieren und Dekodieren erhöhtes Fehlerpotential.\footnote{vgl. \cite{rothe}, Seite 10f}

% subsection kommunikationsmodell (end)

\subsection{Anzahl der Teilnehmer} % (fold)
\label{sub:anzahl_der_teilnehmer}
1 zu 1 vs. 1 zu viele vs. viele zu viele Kommunikation
% subsection anzahl_der_teilnehmer (end)

\subsection{Synchronität} % (fold)
\label{sub:synchronitat}
Synchrone vs. Asynchrone Kommunikation
% subsection synchronitat (end)

\subsection{Elemente des Web 2.0} % (fold)
\label{sub:elemente_des_web_2_0}
Interaktiv, jeder „produziert“, keine Zuschauer
% subsection elemente_des_web_2_0 (end)

\subsection{Haftung im Web 2.0} % (fold)
\label{sub:haftung_im_web_2_0}

% subsection haftung_im_web_2_0 (end)

\subsection{Semantik} % (fold)
\label{sub:semantik}
Web DREI NULL!!!
% subsection semantik (end)

\subsection{Forensysteme} % (fold)
\label{sub:forensysteme}

% subsection forensysteme (end)

\subsection{Wikis} % (fold)
\label{sub:wikis}

% subsection wikis (end)

\subsection{Knowledge--Bases} % (fold)
\label{sub:knowledge_bases}

% subsection knowledge_bases (end)
% section kommunikationsformen_im_internet (end)

\section{Online Campus Portal} % (fold)
\label{sec:online_campus_portal}

\subsection{Beschreibung des OCP} % (fold)
\label{sub:beschreibung_des_ocp}

% subsection beschreibung_des_ocp (end)

\subsection{Benötigte Schnittstellen} % (fold)
\label{sub:benotigte_schnittstellen}
z.B. Benutzerverwaltung
% subsection benotigte_schnittstellen (end)

\subsection{Integration mit vorhandenen Systemen} % (fold)
\label{sub:integration_mit_vorhandenen_systemen}
Designelemente \& Benutzerführung
% subsection integration_mit_vorhandenen_systemen (end)

% section online_campus_portal (end)

\section{Ziele} % (fold)
\label{sec:ziele}

\subsection{Zielgruppe} % (fold)
\label{sub:zielgruppe}

% subsection zielgruppe (end)

\subsection{Organisatorische Ziele} % (fold)
\label{sub:organisatorische_ziele}

% subsection organisatorische_ziele (end)

\subsection{Pädagogische Ziele} % (fold)
\label{sub:padagogische_ziele}
z.B. Studierende helfen sich selbst, Problem des Monats, ...
% subsection padagogische_ziele (end)

\subsection{Reichweite} % (fold)
\label{sub:reichweite}
Öffentlich? Hochschulöffentlich? Nur für Studierende? Nur für Lehrkräfte? Privat?
% subsection reichweite (end)

\subsection{Inhaltliche Qualität} % (fold)
\label{sub:inhaltliche_qualitat}

% subsection inhaltliche_qualitat (end)

\subsection{Zielkonflikt: Betreuungsqualität vs. Involvement} % (fold)
\label{sub:zielkonflikt_betreuungsqualitat_vs_involvement}
Der wohl größte Konflikt besteht wohl zwischen dem Anspruch, Anfragen der Studierenden möglichst schnell und kompetent zu bearbeiten und der Absicht, möglichst viele Studierende in die Lösungsfindung bei akademischen Fragen zu involvieren. Hier bedarf es ein System, nach dem Wichtigkeit, Dringlichkeit und pädagogisches Potential der Fragen zu bewerten sind, damit vom Dozenten entsprechend dieser Einordnung eine passende Antwortstrategie gewählt werden kann. 

%%%%%%%%%%%%%%%%%%%%%%%%%%%%%%%%%%%%%%%%%%%%%%%%%%%%%%%%%%%%%%%
%
% Welcome to Overleaf --- just edit your LaTeX on the left,
% and we'll compile it for you on the right. If you give
% someone the link to this page, they can edit at the same
% time. See the help menu above for more info. Enjoy!
%
% Note: you can export the pdf to see the result at full
% resolution.
%
%%%%%%%%%%%%%%%%%%%%%%%%%%%%%%%%%%%%%%%%%%%%%%%%%%%%%%%%%%%%%%%
% :Author: Stefan Kottwitz
% :Source: TeXblog - TikZ: shaded cube
%          http://texblog.net/latex-archive/graphics/tikz-cube-3d/
\begin{comment}

:Title: Sudoku 3D cube
:Tags: 3D, Transformations

This example shows how to create an effective 3D effect using the ``slant`` transformation.
Shading has been added to enhance the 3D impression. Read more about this example over
at TeXblog_. 

.. _TeXblog: http://texblog.net/latex-archive/graphics/tikz-cube-3d/

:Author: Stefan Kottwitz
:Source: `TeXblog - TikZ: shaded cube`__

.. __: http://texblog.net/latex-archive/graphics/tikz-cube-3d/

\end{comment}
\begin{figure}[H]
\begin{center}
\begin{tikzpicture}[every node/.style={minimum size=1cm},on grid, scale=3]
\begin{scope}[every node/.append style={yslant=-0.5},yslant=-0.5]
%  \shade[right color=gray!10, left color=black!50] (0,0) rectangle +(2,2);
  \node at (0.5,1.6) {geringe};
  \node at (0.5,1.4) {Relevanz};
  \node at (1.5,1.6) {hohe};
  \node at (1.5,1.4) {Relevanz};
  \node at (0.5,0.6) {geringe};
  \node at (0.5,0.4) {Relevanz};
  \node at (1.5,0.6) {hohe};
  \node at (1.5,0.4) {Relevanz};
  \draw (0,0) grid (2,2);
\end{scope}
\begin{scope}[every node/.append style={yslant=0.5},yslant=0.5]
%  \shade[right color=gray!70,left color=gray!10] (2,-2) rectangle +(2,2);
  \node at (2.5,-0.4) {hohes};
  \node at (2.5,-0.6) {Potential};
  \node at (3.5,-0.4) {hohes};
  \node at (3.5,-0.6) {Potential};
  \node at (2.5,-1.4) {geringes};
  \node at (2.5,-1.6) {Potential};
  \node at (3.5,-1.4) {geringes};
  \node at (3.5,-1.6) {Potential};
  \draw (2,-2) grid (4,0);
\end{scope}
\begin{scope}[every node/.append style={
    yslant=0.5,xslant=-1},yslant=0.5,xslant=-1
  ]
 % \shade[bottom color=gray!10, top color=black!80] (4,2) rectangle +(-2,-2);
  \node at (2.5,1.6) {hohe};
  \node at (2.5,1.4) {Dringlichkeit};
  \node at (2.5,0.6) {hohe};
  \node at (2.5,0.4) {Dringlichkeit};
  \node at (3.5,1.6) {geringe};
  \node at (3.5,1.4) {Dringlichkeit};
  \node at (3.5,0.6) {geringe};
  \node at (3.5,0.4) {Dringlichkeit};
  \draw (2,0) grid (4,2);
\end{scope}
\end{tikzpicture}
\caption{Einteilung der Forenbeiträge}
\label{fig:wuerfel}
\end{center}
\end{figure}


% subsection zielkonflikt_betreuungsqualitat_vs_involvement (end)

% section ziele (end)

\section{Technische und wirtschaftliche Aspekte} % (fold)
\label{sec:technische_und_wirtschaftliche_aspekte}

\subsection{Benötige Hard- und Software} % (fold)
\label{sub:benotige_hard_und_software}

% subsection benotige_hard_und_software (end)

\subsection{Benötigtes Budget} % (fold)
\label{sub:benotigtes_budget}

% subsection benotigtes_budget (end)

% section technische_und_wirtschaftliche_aspekte (end)

\section{Best Practice \& Marktanalyse} % (fold)
\label{sec:best_practice}

\subsection{erp4students} % (fold)
\label{sub:erp4students}
Superschnelle Dozenten --- wenig Austausch zwischen Studierenden
% subsection erp4students (end)

\subsection{shootcamp.at} % (fold)
\label{sub:shootcamp_at}
Interaktion zwischen Teilnehmern wird gefördert, sparsamer Einsatz von Dozentenmeinung.\\
Gute Uploadmöglichkeiten für Bilder
% subsection shootcamp_at (end)

\subsection{OnCampus.de} % (fold)
\label{sub:oncampus_de}
Viele Dateianhänge (Beiträge in .DOC) behindern den Austausch
% subsection oncampus_de (end)

\subsection{Incentives} % (fold)
\label{sub:infentives}
shootcamp.at --- like\\
Fernstudenten.de --- Status nach Beitragszahl\\
StackOverflow.com --- Punktesystem
% subsection infentives (end)

% section best_practice (end)

\section{SWOT-Analysen} % (fold)
\label{sec:swot_analysen}
Je nach  Themenfeld, evt. auch in die anderen Kapitel integriert
% section swot_analysen (end)