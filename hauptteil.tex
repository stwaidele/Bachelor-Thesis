\section{Plazierung am Markt} % (fold)
\label{sec:plazierung_am_markt}

\subsection{Marktanalyse: Welche Communities gibt es schon?} % (fold)
\label{sub:marktanalyse_welche_communities_gibt_es_schon}
Für Studierende an der AKAD–University gibt es eine Reihe von Angeboten im Internet, die einzelne Elemente der in Kapitel~\myref{sub:zusammenspiel_der_werkzeuge} beschriebenen \ac{cscl}–Umgebung abdecken. Diese werden zum Teil bereits heute von der AKAD–University bereitgestellt, wie etwa die Materialien zum Selbststudium, Lernvideos und eine Diskussionsfunktion auf Ebene der Studienmodule. Letztere bietet allerdings nicht die für ein Forum notwendige Übersichtlichkeit und eignet sich somit nicht als Ersatz für ein Forum.\footnote{vgl. \cite{defstud}, Abschnitt „Moduldiskussion / Studienbetreuung“}

Zum Austausch zwischen den Studierenden stehen mit \buzz{Fernstudenten.de}\footnote{Öffentliche Beiträge. Zum Einstellen von Inhalten ist ein Benutzerkonto notwendig. Die Hochschulzugehörigkeit wird jedoch nicht geprüft. Siehe \url{Fernstudenten.de}} und einem Netzwerk an \buzz{Facebook–Gruppen}\footnote{Beiträge nur für Gruppenmitglieder einsehbar. Die Hochschulzugehörigkeit wird oberflächlich geprüft. Dozenten und Mitarbeiter der Hochschule werden nicht zugelassen. Siehe \url{facebook.com/groups/AKADStudentengruppe/}} unabhängige Angebote zur Verfügung.

Bei Fernstudenten.de werden neben Fragen zum Studienablauf auch Gedächtnisprotokolle von Klausurstellungen und von Studierenden erstellte Zusammenfassungen des Lernstoffs oder andere Lernhilfen ausgetauscht. Der Charakter der Kommunikation ist hierbei asynchron mit langen Beiträgen und Diskussionen die sich über Tage und Wochen hinziehen. In den Facebookgruppen werden neben aktuellen Fragen zum Studienablauf auch Hinweise auf Zeitschriftenartikel, Lernmaterialien und anderes veröffentlicht. Der Charakter der Kommunikation ist hierbei eher synchron mit meist kurzen Beiträgen und Diskussionen die meist nach einigen Stunden oder innerhalb weniger Tage abgeschlossen sind.
Die Studierendenvertretung nutzt sowohl Fernstudenten.de als auch die Facebookgruppen als Kontaktmedium zu den Studierenden.

Audio– und Videovorlesungen sowie weitere Lernmaterialien stehen online von vielen Anbietern\footnote{z.B. die im \buzz{\ac{OEC}} oder auch bei \buzz{iTunes U} zusammengeschlossenen Universitäten} sowohl auf eigenen Plattformen\footnote{z.B. Hasso Plattner Institut (\url{open.hpi.de}) oder Harvard (\url{cs50.harvard.edu/})} sowie auf Videoportalen\footnote{z.B. YouTube (\url{YouTube.com}) oder Vimeo (\url{Vimeo.de})} zur Verfügung.
% subsection marktanalyse_welche_communities_gibt_es_schon (end)

\subsection{Alleinstellungsmerkmale \& Plazierung} % (fold)
\label{sub:alleinstellungsmerkmale_plazierung}
Ein Alleinstellungsmerkmal der AKAD–University ist das Angebot von anerkannten akademischen Abschlüssen bei individueller und absolut freier Festlegung der Lernzeiten durch den Studierenden. Auch die Prüfungsleistungen sind im Rahmen der angebotenen Termine frei wählbar. Eine Beschränkung der Studienzeit besteht laut \ac{SPO} nicht. Konkurrenzangebote sind entweder an Semester gebunden\footnote{z.B. \ac{LINAVO} oder ERP4students} oder die Abschlüsse sind nicht staatlich oder zumindest in der Wirtschaft allgemein anerkannt.

Durch die Ausrichtung des Lernangebots auf eine anerkannte \ac{SPO} sowie durch die Betreuung des Lernvorgangs durch hochschuleigene Dozenten wird eine entsprechende Lernplattform trotz alternativer Angebote von den Studierenden genutzt werden. Die \ac{CSCL}–Plattform wird als integraler Bestandteil des Studienangebots und der Leistung der AKAD–University wahrgenommen werden und zentrale Aufgaben im akademischen und organisatorischen Bereich unterstützen.
% subsection alleinstellungsmerkmale_plazierung (end)

% section plazierung_am_markt (end)

\section{Einbindung in die Organisation} % (fold)
\label{sec:einbindung_in_die_organisation}

\subsection{Forum als Kommunikationsmedium der Betreuung} % (fold)
\label{sub:forum_als_kommunikationsmedium_der_betreuung}

% subsection forum_als_kommunikationsmedium_der_betreuung (end)

\subsection{Nutzung in der Studierendenvertretung} % (fold)
\label{sub:nutzung_in_der_studierendenvertretung}

% subsection nutzung_in_der_studierendenvertretung (end)

% section einbindung_in_die_organisation (end)

\section{Struktur} % (fold)
\label{sec:struktur}

\subsection{Nutzung \& Reichweite} % (fold)
\label{sub:reichweite}
Öffentlich? Hochschulöffentlich? Nur für Studierende? Nur für Lehrkräfte? Privat?
% subsection reichweite (end)

\subsection{Rechtesystem} % (fold)
\label{sub:rechtesystem}

% subsection rechtesystem (end)

\subsection{Themengliederung} % (fold)
\label{sub:themengliederung}

% subsection themengliederung (end)

% section struktur (end)

\section{Qualitätssicherung} % (fold)
\label{sec:qualitatssicherung}

\subsection{Incentive-System} % (fold)
\label{sub:incentive_system}
Like, Punkte, Status, etc.

z.B. Stackoverflow, Facebook, Fernstudenten, Shootcamp
% subsection incentive_system (end)

\subsection{Meldesystem} % (fold)
\label{sub:meldesystem}

% subsection meldesystem (end)

\subsection{Prüfung auf Plagiate} % (fold)
\label{sub:prufung_auf_plagiate}

% subsection prufung_auf_plagiate (end)

% section qualitatssicherung (end)

\section{Redaktionelles Konzept} % (fold)
\label{sec:redaktionelles_konzept}
\subsection{Forum als Werkzeug für Dozenten} % (fold)
\label{sub:forum_als_werkzeug_fur_dozenten}
z.B. LinuxBasics.org
% subsection forum_als_werkzeug_fur_dozenten (end)

\subsection{Forum als Werkzeug für Studierende} % (fold)
\label{sub:forum_als_werkzeug_fur_studierende}
z.B. Themenforen nach Interesse
% subsection forum_als_werkzeug_fur_studierende (end)

\subsection{Aufgabe des Monats} % (fold)
\label{sub:aufgabe_des_monats}

„Stellt euch mal eine \textbf{gemeine} Aufgabe zu …“
% subsection aufgabe_des_monats (end)

\subsection{Übernahme von Themen in die Knowledge--Base} % (fold)
\label{sub:ubernahme_von_themen_in_die_knowledge_base}

% subsection ubernahme_von_themen_in_die_knowledge_base (end)

\subsection{Wiki--Seiten zur Zusammenarbeit} % (fold)
\label{sub:wiki_seiten_zur_zusammenarbeit}

% subsection wiki_seiten_zur_zusammenarbeit (end)

% section redaktionelles_konzept (end)

\section{Controlling} % (fold)
\label{sec:controlling}
Wie kann die Zielerreichung gemessen werden?
% section controlling (end)